\documentclass[a4paper,10pt]{article}

\begin{document}

\title{%
  Request Coin \\
  \medbreak
  \small  Open-Source, Decentralized Oracle and Uptime Ledger\\
     }

\date{\today}
\author{Austin Thomas Griffith}
\maketitle


\begin{center}
\textbf{Abstract}
\end{center}
decentralized oracle \\
uptime ledger \\
http to ipfs \\

what it is \\
how it works \\


\begin{center}
\textbf{Controllers}
\end{center}

Head/Main \\
Provides a mapping to all the different active contracts. \\
This lets you upgrade a contract by deploying a new version, copying state, and updating head contract with new address \\

Auth \\
An auth contract that will store a mapping of addresses to an ENUM permission \\
Methods to get and set permission based on role of sender \\
Ability for trusted addresses to vote out or in new trusties? \\

Token \\
An ERC20 or similar token contract that will be used as the currency for requests \\
Even the token could theoretically be upgraded (freeze contract, transfer balances, set new address for token contract in head) \\

Requests \\
An open job queue of Request still waiting for more miners \\
Requests will be represented by a struct and can have many different forms \\
(Some requests will want consensus while others will want single calls)

Processors \\
Once a request is fulfilled by a miner, it should be processed on the way back in \\
Meaning, the request struct will define a contract ENUM (looked up on head) to process the results \\
Processors could range from simple passalong, to hash, to different forms on consensus \\

Responses \\
Stores references and other meta data for responses \\
This will include the original request id, status code, request time, and content hash \\
This could also store a consensus or anything else the processor ends up collecting and specifying \\


\begin{center}
\textbf{Request Response Lifecycle }
\end{center}
End user sends Request struct to Requests contract along with some amount of RQC token to incentivize miners \\
Miners read Requests Events, pick the best job (incentives), and make http requests \\
Depending on request specifications, the Miner will at least submit the a response to the processor defined in the request \\
The miner may provide hash of the content, the response code, the response time, an IPFS address of response, location of miner, etc to the processor
The processor will accumulate responses until the defined consensus is met and create a response \\
This response event is watched either on chain or off by either RequestCoin clients or third party apps \\
Public and decentralized history of all requests/consensus/responses \\


\begin{center}
\textbf{Todo }
\end{center}
Register requestcoin.eth, request.eth, and rqc.eth \\

\end{document}
